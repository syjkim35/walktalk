\documentclass[fleqn,12pt]{article}

\usepackage[english]{babel}
\usepackage{marginnote}
\usepackage[margin=1in]{geometry}
\usepackage{amsmath,amssymb}
\usepackage{array}
\usepackage{enumerate}
\usepackage{color}
\usepackage[usenames,dvipsnames]{xcolor}
\usepackage{tikz}
\usepackage{hyperref}
\usepackage{longtable}
\usepackage{fancyvrb}

\usepackage{fancyhdr}
\pagestyle{fancy}

\fancyhead{}
\fancyhead[L]{Walk with Talk\\API Specification}
% \fancyhead[R]{\texttt{cs70-iv}\\SID 25774340}
\setlength{\headheight}{27.2pt}

\renewcommand{\headrulewidth}{0.2pt}
\renewcommand{\arraystretch}{1.5}

\newcommand{\nth}[2]{$ \text{#1}^{\text{#2}} $}
\newcommand{\sub}[2]{$ \text{#1}_{\text{#2}} $}

\title{\textsc{Walk With Talk}}
\author{API Specification}
\date{Last Updated: \today}

\newcommand{\g}{\texttt{GET}}
\newcommand{\p}{\texttt{POST}}
\renewcommand{\jsnull}{\texttt{null}}

\definecolor{darkgreen}{rgb}{0,0.5,0}

\begin{document}
\maketitle

On error, end-points will return a JSON string describing the error in the
following format:
\begin{Verbatim}[commandchars=\\\{\}]
  "error": "The error message",
  "code":  200,   \textcolor{darkgreen}{// An HTTP error code}
  "field": "The field that caused the error"
\end{Verbatim}
\noindent
If the error is generic, that is, if the error is not restricted to a certain
field (such as a duplicate username), all fields should be highlighted and
the \texttt{"field"} key will contain \jsnull. A full list of HTTP error code
and their uses can be found
\href{https://en.wikipedia.org/wiki/List_of_HTTP_status_codes}{here}.
\\\\
\begin{tabular}{@{} >{\ttfamily/}l |c| p{5in}}
\hline
URL & Methods & Description \\
\hline
login & \g, \p &
Logs in a user. \newline
The \g{} action retrieves current user details. Results are returned in JSON
format with the following fields:
{\begin{verbatim}
  "username",
  "sex",
  "age"
\end{verbatim}}
\newline
The same fields are expected during the \p{} action, in addition to password
data in the \texttt{"password"} field. This is a SHA-256 hash of the data
entered by the user into the form field.
\newline\newline
The submission will return a 401 if the data does not match an existing user,
with a \jsnull{} value for \texttt{"field"}. On success, there will be a code
200, as well as JSON data just like in the \g{} request.
\\
\hline
register & \p &
Registers a new user. \newline
The \p{} action will expect the same fields as \texttt{/login}, in addition to
a \texttt{"password\_confirm"} field that follows the same characteristics as
\texttt{"password"}.
\newline\newline
The submission will return a

\end{tabular}
\end{document}
